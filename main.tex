\documentclass{resume}

\usepackage{hyperref}
\usepackage{kotex}

\setmainhangulfont{Noto Serif CJK KR}[
  UprightFont=* Light, BoldFont=* Bold,
  Script=Hangul, Language=Korean, AutoFakeSlant,
]
\setsanshangulfont{Noto Sans CJK KR}[
  UprightFont=* DemiLight, BoldFont=* Medium,
  Script=Hangul, Language=Korean
]
\setmathhangulfont{Noto Sans CJK KR}[
  SizeFeatures={
    {Size=-6,  Font=* Medium},
    {Size=6-9, Font=*},
    {Size=9-,  Font=* DemiLight},
  },
  Script=Hangul, Language=Korean
]

\name{Ryang Sohn (손량)}
\email{\href{mailto:ryangsohn@postech.ac.kr}{ryangsohn [at] postech [dot] ac [dot] kr}}

\begin{document}

\begin{res-section}{Education}
  \begin{res-subsection}{POSTECH (Pohang University of Science and Technology)}{Feb 2022 -- Ongoing}
    Majoring in Computer Science and Engineering, double major in Mathematics (GPA: 4.16/4.30)
  \end{res-subsection}
\end{res-section}

\begin{res-section}{Skills}
  \begin{tabular}{p{0.25\linewidth}p{0.7\linewidth}}
    Programming Languages
      & Python, Rust, C/C++, JavaScript, Java, Go \\
    Tooling
      & Git/GitHub, Docker, Linux, CMake \\
    Natural Languages
      & Korean (native), English (working proficiency)
  \end{tabular}
\end{res-section}

\begin{res-section}{Work Experience}
  \begin{res-subsection}{Theori}{Jan 2024 -- Ongoing}
    ChainLight WARD (Web3 Automated Risk Detection) Intern Researcher

    \item Skills used: Rust, Static Program Analysis, Solidity

    \item Worked on security-focused static analysis engine for web3 applications.

    \item Areas of interest:
    \vspace{-0.5em}
    \begin{list}{--}{}
      \itemsep -0.5em
      \item Translating Solidity code to intermediate representation suitable for data-flow analysis.
      \item Simplified memory model of Ethereum Virtual Machine.
      \item Vulnerability detection of smart contracts using data-flow analysis.
    \end{list}
  \end{res-subsection}

  \begin{res-subsection}{PoApper Inc.}{Jan 2022 -- Mar 2023}
    Part-time Backend Engineer

    \item Skills used: Python, Go

    \item Developed \texttt{fights.ai}, an environment for competitive game-playing agents.

    \item Areas of interest:
    \vspace{-0.5em}
    \begin{list}{--}{}
      \itemsep -0.5em
      \item Developer-friendly Python API for game-playing agents.
      \item Server infrastructure based on message queue for competitive gameplay.
      \item Isolated per-agent environment for multiplayer games.
    \end{list}
  \end{res-subsection}
\end{res-section}

\begin{res-section}{Awards and Honors}
  \begin{res-subsection}{National Science \& Technology Scholarship}{May 2024}
    Endorsed by the university in the 3rd year.
  \end{res-subsection}

  \begin{res-subsection}{POSTECH CSE Global Leadership Program 2024 Spring}{Mar 2024}
    A scholarship program for high-performing POSTECH CSE students.
  \end{res-subsection}

  \begin{res-subsection}{POSTECH CSE Global Leadership Program 2023 Fall}{Sep 2023}
    A scholarship program for high-performing POSTECH CSE students.
  \end{res-subsection}

  \begin{res-subsection}{Crypto Contest 2022, 2nd Prize}{Oct 2022}
    Cryptanalysis competition hosted by South Korean Ministry of Defense.

    \item Contributions:
    \vspace{-0.5em}
    \begin{list}{--}{}
      \itemsep -0.5em
      \item Multi-threaded PoC code for attacking weak Bitcoin-like wallet scheme.
      \item Security analysis of Sponge-based hash function.
    \end{list}
  \end{res-subsection}

  \begin{res-subsection}{POSTECH Programming Contest 2022, Freshman Prize}{Sep 2022}
    Coding competition for POSTECH students.

    \item Participated as Team 대줴패, ranked first among freshman students.
  \end{res-subsection}

  \begin{res-subsection}{Artificial Intelligence Accelerator Design Competition, Encouragement Prize}{Jun 2022}
    Competition to design FPGA-based accelerator for neural networks.

    \item Worked on: 8-bit quantization algorithm of YOLOv3 neural network.
  \end{res-subsection}
\end{res-section}

\begin{res-section}{Personal Projects}
  \begin{res-subsection}{\texttt{stapl} -- Simple, Type-Annotated Programming Language}{}
    A compiler for imperative programming language with type annotations.

    \item Written in C++ and based on LLVM.

    \item Striving to follow best practices of modern C++ and software development (modularity, unit testing, documentation, etc.)
  \end{res-subsection}

  \begin{res-subsection}{GPU-accelerated Ray Tracer}{}
    Developed a ray tracer that can simulate various materials.

    \item Started as a term-project for Computer Graphics course.

    \item Written in Vulkan compute shader, more than 10x speedup compared to CPU based ray tracer.

    \item Learned internals and low-level details of modern graphics pipeline.

    \item Notable features:
    \vspace{-0.5em}
    \begin{list}{--}{}
      \itemsep -0.5em
      \item Support for metallic, dielectric, and diffuse materials.
      \item Simulation for hypothetical portal materials.
      \item Utilization of modern technology stack, such as Vulkan and dynamic rendering.
    \end{list}
  \end{res-subsection}

  \begin{res-subsection}{TML (Tiny ML) Compiler}{}
    The final programming assignment of Programming Language course.

    \item Compiles Tiny ML, a subset of Standard ML, into machine code for virtual machine \textsf{Mach}.

    \item Written in OCaml.

    \item Learned internals of compilers for functional languages, and various implementation strategies.

    \item Notable features:
    \vspace{-0.5em}
    \begin{list}{--}{}
      \itemsep -0.5em
      \item SSA-like intermediate representation.
      \item Support for closures, high-order functions, recursive datatypes and pattern matching.
    \end{list}
  \end{res-subsection}

  \begin{res-subsection}{PintOS Implementation}{}
    Implementing PintOS, an educational operating system.

    \item Written in C and x86 assembly.

    \item Worked on threading, userspace programs, and virtual memory system.

    \item Notable features:
    \vspace{-0.5em}
    \begin{list}{--}{}
      \itemsep -0.5em
      \item Virtual memory system similar to object-based reverse mapping of Linux.
      \item Reduced memory usage by reusing already-loaded code sections.
    \end{list}
  \end{res-subsection}
\end{res-section}

\end{document}
