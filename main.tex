\documentclass{resume}

\usepackage{kotex}

\setmainhangulfont{Noto Serif CJK KR}[
  UprightFont=* Light, BoldFont=* Bold,
  Script=Hangul, Language=Korean, AutoFakeSlant,
]
\setsanshangulfont{Noto Sans CJK KR}[
  UprightFont=* DemiLight, BoldFont=* Medium,
  Script=Hangul, Language=Korean
]
\setmathhangulfont{Noto Sans CJK KR}[
  SizeFeatures={
    {Size=-6,  Font=* Medium},
    {Size=6-9, Font=*},
    {Size=9-,  Font=* DemiLight},
  },
  Script=Hangul, Language=Korean
]

\name{Ryang Sohn}
\email{loop.infinitely@gmail.com}

\begin{document}

\begin{res-section}{Education}
  \begin{res-subsection}{POSTECH (Pohang University of Science and Technology)}{Feb 2022 - Ongoing}
    Majoring in Computer Science and Engineering, double major in Mathematics
  \end{res-subsection}
\end{res-section}

\begin{res-section}{Skills}
  \begin{tabular}{p{0.25\linewidth}p{0.7\linewidth}}
    Programming Languages
      & Python, Rust, C/C++, JavaScript, Java, Go \\
    Tooling
      & Git/GitHub, Docker, Linux, CMake \\
    Natural Languages
      & Korean (native), English (working proficiency)
  \end{tabular}
\end{res-section}

\begin{res-section}{Work Experience}
  \begin{res-subsection}{PoApper Inc.}{Jan 2022 -- Mar 2023}
    Part-time Backend Engineer

    \item Skills used: Python, Go

    \item Developed \texttt{fights.ai}, an environment for competitive game-playing agents
  \end{res-subsection}
\end{res-section}

\begin{res-section}{Awards and Honors}
  \begin{res-subsection}{POSTECH CSE Global Leadership Program 2023}{Sep 2023}
    A scholarship program for high-performing POSTECH CSE students.
  \end{res-subsection}

  \begin{res-subsection}{2022 암호분석경진대회 국방정보본부장상}{Oct 2022}
    Cryptanalysis contest held by Ministry of Defense of Korea.

    \item Worked on: multi-threaded PoC code for attacking weak Bitcoin wallet, security anlaysis of Sponge-based hash function.
  \end{res-subsection}

  \begin{res-subsection}{POSTECH Programming Contest 2022 신인상}{Sep 2022}
    Coding competition for POSTECH students.

    \item Participated as Team 대줴패, ranked first among freshman students.
  \end{res-subsection}

  \begin{res-subsection}{2022 Deep Learning Hardware 설계 경진대회 장려상}{Jun 2022}
    Competition to design FPGA-based accelerator for neural networks.

    \item Worked on: 8-bit quantization algorithm of YOLOv3 neural network
  \end{res-subsection}
\end{res-section}

\begin{res-section}{Personal Projects}
  \begin{res-subsection}{\texttt{stapl} -- Simple, Type-Annotated Programming Language}{}
    A compiler for imperative programming language with type annotations.

    \item Written in C++ and based on LLVM.

    \item Striving to follow best practices of modern C++ and software development (modularity, unit testing, documentations, etc.)
  \end{res-subsection}

  \begin{res-subsection}{PintOS Implementation}{}
    Implementing PintOS, an educational operating system.

    \item Worked on threading, userspace programs, and vitual memory system similar to object-based reverse mapping of Linux.
  \end{res-subsection}
\end{res-section}

\end{document}
